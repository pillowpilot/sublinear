\documentclass[10pt,a4paper]{article}
\usepackage[utf8]{inputenc}
\usepackage[english, spanish]{babel}
\usepackage{amsmath, amsfonts, amssymb, amsthm}
%\usepackage[backend=biber]{biblatex}
%\bibliography{biblio.bib}

\usepackage{graphicx}

% Asymptotic Notations
\renewcommand{\O}[1]{$\mathcal{O}(#1)$}
\renewcommand{\o}[1]{$o(#1)$}

% Definition of Theorem Enviroments
\newtheorem{definition}{Definición}[part]

\author{Carlos Federico Gaona}
\title{Resumen de Algoritmos Sublineales}
\date{}

\begin{document}
\maketitle
%TODO Introduction to problem solving

Intuitivamente podemos sugerir que si no disponemos de \O{n} de algún recurso y, sin embargo, necesitamos de al menos \O{n} de el mismo recurso para presentar la solución óptima, entonces no podríamos solucionar el problema. Mas precisamente, no podremos responder de forma \textbf{óptima} y con \textbf{certeza}. Para ello podríamos presentar una solución ``cercana'' a la solución óptima.

% TODO Improve definition of Heuristics, Aproximation and Probabilistic Algorithms.
La manera de satisfacer esta ``cercanía'' entre soluciones óptimas y soluciones retornadas por un algoritmo es representada como límites estrictamente definidos en los Algoritmos Aproximados, como la respuesta a alguna estrategia en los Algoritmos Heurísticos o como la media de múltiples soluciones en los Algoritmos Probabilísticos.

% TODO Graphics about those three types of algorithms

\part{Conceptos}
\section{Definición de Algoritmos Sublineales}
En el sentido clásico los algoritmos presentan como salida la solución óptima al problema que resuelven, sin embargo los algoritmos aproximados presentan como salida una \textbf{solución no necesariamente óptima confinado dentro unos límites} alrededor de la solución óptima. Los algoritmos sublineales presentan como salida una \textbf{solución no necesariamente óptima confinado dentro de unos límites} alrededor de la solución óptima \textbf{con una probabilidad definida}.

Formalmente, son una rama de los $(1+\epsilon, \delta)$\textit{-approximate algorithms} donde utilizan \o{n} de tiempo, espacio o comunicaciones y respectivamente son llamados como \textit{property testing algorithms}, \textit{data stream algorithms} y \textit{communication complexity algorithms}.

De ahora en adelante podremos referirnos a los Algoritmos Sublineales simplemente con AS.

\section{Property Testing}
La verificación de propiedades, o \textit{property testing} en inglés, sobre una entrada $f$ de tamaño $n$ cuando esta es demasiado grande para ser procesada en \O{n}, ya sea por el valor mismo de $n$ o por las limitaciones inherentes del problema, es una de las aplicaciones de los AS.

\begin{definition}
  Dado una entrada, definida por la función $f:\mathcal{D} \rightarrow \mathcal{F}$, se dice que es $\epsilon$-cercano de satisfacer una propiedad P si existe una función $f':\mathcal{D} \rightarrow \mathcal{F}$ tal que difiere de $f$ en no más de $\epsilon|\mathcal{D}|$ lugares y satisface P. Si una función no es $\epsilon$-cercano de satisfacer una propiedad P se dice que es $\epsilon$-lejano de satisfacer una propiedad P.
\end{definition}

\end{document}
